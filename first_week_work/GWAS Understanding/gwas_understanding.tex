\documentclass{article}
\usepackage{amsmath}
\usepackage{amsfonts}
\usepackage{setspace}
\usepackage{graphicx}
\usepackage{subcaption}
\usepackage{tikz}
\usepackage{wrapfig}
\usepackage{caption}
\usepackage{titling}
\usepackage{a4wide}
\usepackage{lipsum}
\usepackage{tikzpagenodes}
\usepackage{hyperref}
\usepackage{fancyhdr} % Added package

\begin{document}
\title{Patrick GWAS notes}
%\author{\vspace{-10mm}0451 972 879 | patrick.m.gibbs@protonmail.com}
\setlength{\droptitle}{-12em}
\date{}
\maketitle
\vspace*{-20mm}
\setlength{\parindent}{0pt}
\noindent\textbf{Linear Fixed Effects Model}
\vspace{2mm}

Test each marker independently with the following model:


\begin{equation*}
y = \beta_0 + \beta x
\end{equation*}

Find the probability $p$ that $\beta \neq 0$ 
\newline

To do this we fit each model using


\[
\begin{bmatrix}
\beta_0\\
\beta\end{bmatrix}
=(X^T X)^{-1} X^T y\]

For each $\beta$ we find the t-statisitc computed as:

\begin{equation*}
t = \frac{\beta}{se_b} 
\end{equation*}

where $\beta$ is as computed above, and $se_\beta$ is the standard error of beta computed as:

\begin{equation*}
se_\beta = \frac{\sqrt{\frac{\sum_i (y_i - \hat{y}_i)^2} {n-2}}}{\sum_j (x_j-\bar{x})}
\end{equation*}

The $p$ can then be computed as:

\begin{equation*}
p=2\cdot(1-F(t))
\end{equation*}

where $F$ is the t-distribution CDF with $n-2$ degrees of freedom. $p$ values and then $-\log$ transformed to produce a manhattan plot.
\newline

\noindent\textbf{Mixed Effects Model}



\end{document}
